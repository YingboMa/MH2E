\documentclass[8pt,twocolumn]{article}
%\usepackage{amsmath,amssymb,amsthm,amsbsy,amsfonts,mathtools}
\usepackage{amsmath}
\usepackage{amssymb}
\usepackage{amsthm}
\usepackage{physics}
\usepackage{hyperref}
\usepackage{exercise}
\usepackage[makeroom]{cancel}
\usepackage[margin=2em]{geometry}

\usepackage{graphicx}

\usepackage{footmisc}
\DefineFNsymbols{mySymbols}{{\ensuremath\dagger}{\ensuremath\ddagger}\S\P
   *{**}{\ensuremath{\dagger\dagger}}{\ensuremath{\ddagger\ddagger}}}
\setfnsymbol{mySymbols}

\newcommand{\N}{\mathbb{N}}
\newcommand{\R}{\mathbb{R}}
\newcommand{\Z}{\mathbb{Z}}
\newcommand{\Q}{\mathbb{Q}}

\newenvironment{theorem}[2][Theorem]{\begin{trivlist}
\item[\hskip \labelsep {\bfseries #1}\hskip \labelsep {\bfseries #2.}]}{\end{trivlist}}
\newenvironment{lemma}[2][Lemma]{\begin{trivlist}
\item[\hskip \labelsep {\bfseries #1}\hskip \labelsep {\bfseries #2.}]}{\end{trivlist}}
\newenvironment{exercise}[2][Exercise]{\begin{trivlist}
\item[\hskip \labelsep {\bfseries #1}\hskip \labelsep {\bfseries #2.}]}{\end{trivlist}}
\newenvironment{problem}[2][Problem]{\begin{trivlist}
\item[\hskip \labelsep {\bfseries #1}\hskip \labelsep {\bfseries #2.}]}{\end{trivlist}}
\newenvironment{question}[2][Question]{\begin{trivlist}
\item[\hskip \labelsep {\bfseries #1}\hskip \labelsep {\bfseries #2.}]}{\end{trivlist}}
\newenvironment{corollary}[2][Corollary]{\begin{trivlist}
\item[\hskip \labelsep {\bfseries #1}\hskip \labelsep {\bfseries #2.}]}{\end{trivlist}}
\newenvironment{answer}[2][Answer]{\begin{trivlist}
\item[\hskip \labelsep {\bfseries #1}\hskip \labelsep {\bfseries #2.}]}{\end{trivlist}}

\let\emph\relax % there's no \RedeclareTextFontCommand
\DeclareTextFontCommand{\emph}{\bfseries\em}

\setlength{\columnseprule}{0.4pt}
\setlength{\columnsep}{3em}

\usepackage{todonotes}

\author{Yingbo Ma \thanks{Student ID: \tt{16058474}}}
\title{\vspace{-1.cm}Homework 6}
\date{May 20, 2018}

\begin{document}
\maketitle

\begin{Answer}[number=32]
  \emph{16.3.3:}
  Let $A(x,y) = xy+y^2$ and $B(x,y) = x^2+2xy$. We have $\pdv{A}{y} = x+2y$ and
  $\pdv{B}{x} = 2x+2y$. The $\bm$ field is not conservative since
  $\pdv{A}{y}\ne \pdv{B}{y}$.

  \emph{16.3.13:}
  We have $\bm{F} = \grad f = \mqty[x^2y^3\\x^3y^2]$, thus we have
  $f(x,y)=\frac{x^3y^3}{3}+g(y)$ and $\pdv{f}{y} = x^3y^2 + \dv{g}{y} =
  x^3y^2$. Hence, we have $\dv{g}{y} = 0$. If we take $g(y)=0$, then we have
  $f(x,y) = \frac{x^3y^3}{3}$.

  \[\int_C\bm{F}\cdot\dd{r} = \int_C\grad f\cdot\dd{r} = f(-1,3)-f(0,0) = -9.\]

  \emph{16.3.35:}
  Let $P(x,y) = -\frac{y}{x^2+y^2}$, and $Q(x,y)=\frac{x}{x^2+y^2}$, thus we
  have $\pdv{P}{y} = \frac{y^2-x^2}{(x^2+y^2)^2}$, and $\pdv{Q}{x} =
  \frac{y^2-x^2}{(x^2+y^2)^2}$. Hence $\pdv{Q}{x}=\pdv{P}{y}$.

  Let $C_1: x(t) = \cos(t), y(t) = \sin(t), 0\le t\le \pi, C_2:x(t) = \cos(t),
  y(t) = \sin(t), t$ from $0$ to $-\pi$. We have
  \[
    \int_{C_1} \bm{F}\cdot \dd{\bm{r}} = \int_0^\pi \frac{\sin^2t +
    \cos^2t}{\sin^2t + \cos^2t} \dd{t} = \int_0^\pi \dd{t} = \pi,
  \]
  and
  \[
    \int_{C_2} \bm{F}\cdot \dd{\bm{r}} = \int_0^{-\pi}\dd{t} = -\pi.
  \]
  Those integrals are not equal. However, this doesn't contradict the Theorem
  6, since there is a singularity at $(0,0)$ in the function $\bm{F}$.
\end{Answer}
\begin{Answer}[number=33]
  \begin{proof}
    We know that $(x,y,z_1,z_2,z_3)$ is a point in the $5$-dimensional unit
    ball, thus we have the relation $x^2+y^2+z_1^2+z_2^2+z_3^2 = 1^2$. Hence we
    have $z_1^2+z_2^2+z_3^2 = 1^2-x^2-y^2$, which means that the point $(z_1,
    z_2, z_3)$ is exactly in a three dimensional with a radius
    $\sqrt{\abs{1^2-x^2-y^2}}$. This completes the proof.
  \end{proof}
\end{Answer}

\end{document}
