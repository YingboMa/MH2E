\documentclass[8pt,twocolumn]{article}
%\usepackage{amsmath,amssymb,amsthm,amsbsy,amsfonts,mathtools}
\usepackage{amsmath}
\usepackage{amssymb}
\usepackage{amsthm}
\usepackage{physics}
\usepackage{hyperref}
\usepackage{exercise}
\usepackage[makeroom]{cancel}
\usepackage[margin=2em]{geometry}

\usepackage{graphicx}

\usepackage{footmisc}
\DefineFNsymbols{mySymbols}{{\ensuremath\dagger}{\ensuremath\ddagger}\S\P
   *{**}{\ensuremath{\dagger\dagger}}{\ensuremath{\ddagger\ddagger}}}
\setfnsymbol{mySymbols}

\newcommand{\N}{\mathbb{N}}
\newcommand{\R}{\mathbb{R}}
\newcommand{\Z}{\mathbb{Z}}
\newcommand{\Q}{\mathbb{Q}}

\newenvironment{theorem}[2][Theorem]{\begin{trivlist}
\item[\hskip \labelsep {\bfseries #1}\hskip \labelsep {\bfseries #2.}]}{\end{trivlist}}
\newenvironment{lemma}[2][Lemma]{\begin{trivlist}
\item[\hskip \labelsep {\bfseries #1}\hskip \labelsep {\bfseries #2.}]}{\end{trivlist}}
\newenvironment{exercise}[2][Exercise]{\begin{trivlist}
\item[\hskip \labelsep {\bfseries #1}\hskip \labelsep {\bfseries #2.}]}{\end{trivlist}}
\newenvironment{problem}[2][Problem]{\begin{trivlist}
\item[\hskip \labelsep {\bfseries #1}\hskip \labelsep {\bfseries #2.}]}{\end{trivlist}}
\newenvironment{question}[2][Question]{\begin{trivlist}
\item[\hskip \labelsep {\bfseries #1}\hskip \labelsep {\bfseries #2.}]}{\end{trivlist}}
\newenvironment{corollary}[2][Corollary]{\begin{trivlist}
\item[\hskip \labelsep {\bfseries #1}\hskip \labelsep {\bfseries #2.}]}{\end{trivlist}}
\newenvironment{answer}[2][Answer]{\begin{trivlist}
\item[\hskip \labelsep {\bfseries #1}\hskip \labelsep {\bfseries #2.}]}{\end{trivlist}}

\let\emph\relax % there's no \RedeclareTextFontCommand
\DeclareTextFontCommand{\emph}{\bfseries\em}

\setlength{\columnseprule}{0.4pt}
\setlength{\columnsep}{3em}

\usepackage{todonotes}

\author{Yingbo Ma \thanks{Student ID: \tt{16058474}}}
\title{\vspace{-1.cm}Homework 6}
\date{May 20, 2018}

\begin{document}
\maketitle

\begin{Answer}[number=33]
  \emph{16.3.3:}
  Let $A(x,y) = xy+y^2$ and $B(x,y) = x^2+2xy$. We have $\pdv{A}{y} = x+2y$ and
  $\pdv{B}{x} = 2x+2y$. The $\bm$ field is not conservative since
  $\pdv{A}{y}\ne \pdv{B}{y}$.

  \emph{16.3.13:}
  We have $\bm{F} = \grad f = \mqty[x^2y^3\\x^3y^2]$, thus we have
  $f(x,y)=\frac{x^3y^3}{3}+g(y)$ and $\pdv{f}{y} = x^3y^2 + \dv{g}{y} =
  x^3y^2$. Hence, we have $\dv{g}{y} = 0$. If we take $g(y)=0$, then we have
  $f(x,y) = \frac{x^3y^3}{3}$.

  \[\int_C\bm{F}\cdot\dd{r} = \int_C\grad f\cdot\dd{r} = f(-1,3)-f(0,0) = -9.\]

  \emph{16.3.35:}
  Let $P(x,y) = -\frac{y}{x^2+y^2}$, and $Q(x,y)=\frac{x}{x^2+y^2}$, thus we
  have $\pdv{P}{y} = \frac{y^2-x^2}{(x^2+y^2)^2}$, and $\pdv{Q}{x} =
  \frac{y^2-x^2}{(x^2+y^2)^2}$. Hence $\pdv{Q}{x}=\pdv{P}{y}$.

  Let $C_1: x(t) = \cos(t), y(t) = \sin(t), 0\le t\le \pi, C_2:x(t) = \cos(t),
  y(t) = \sin(t), t$ from $0$ to $-\pi$. We have
  \[
    \int_{C_1} \bm{F}\cdot \dd{\bm{r}} = \int_0^\pi \frac{\sin^2t +
    \cos^2t}{\sin^2t + \cos^2t} \dd{t} = \int_0^\pi \dd{t} = \pi,
  \]
  and
  \[
    \int_{C_2} \bm{F}\cdot \dd{\bm{r}} = \int_0^{-\pi}\dd{t} = -\pi.
  \]
  Those integrals are not equal. However, this doesn't contradict the Theorem
  6, since there is a singularity at $(0,0)$ in the function $\bm{F}$.
\end{Answer}

\begin{Answer}[number=34]
  \begin{enumerate}
    \item
      \begin{proof}
        We know that $(x,y,z_1,z_2,z_3)$ is a point in the $5$-dimensional unit
        ball, thus we have the relation $x^2+y^2+z_1^2+z_2^2+z_3^2 \le 1^2$.
        Hence we have $z_1^2+z_2^2+z_3^2 \le 1^2-x^2-y^2$, which means that the
        point $(z_1, z_2, z_3)$ is exactly in a three dimensional with a radius
        $\sqrt{\abs{1^2-x^2-y^2}}$. Also, if the point $(z_1, z_2, z_3)$ is in
        a three dimensional with a radius $\sqrt{\abs{1^2-x^2-y^2}}$, then we
        have $z_1^2+z_2^2+z_3^2 \le 1^2-x^2-y^2$, which naturally leads to
        $x^2+y^2+z_1^2+z_2^2+z_3^2 \le 1^2$, which means that $(z_1, z_2, z_3)$
        point is in the $5$-dimensional unit ball. This completes the proof.
      \end{proof}
    \item
      \begin{proof}
        We have the relation
        \begin{align*}
          V_5&=\int_0^{2\pi} \int_0^R V_{3} \sqrt{1-r^2}\; r \dd{r}\dd{\theta}\\
          V_5&=\frac{2\pi}{5} V_3 = \frac{8\pi^2}{15},
        \end{align*}
        where $V_5$ denotes the volume for a $5$-dimensional unit ball, and
        $V_3$ denotes the volume for a $3$-dimensional unit ball, which is
        $\frac{4}{3}\pi$.
        This completes the proof.
      \end{proof}
  \end{enumerate}
\end{Answer}

\begin{Answer}[number=35]
  \begin{enumerate}
    \item \begin{proof}
      We are going to prove this proposition by direct proof. Using the
      definition of the line integral, we have
      \begin{align*}
        \int_C \bm{F}\cdot \dd{\bm{r}} = \int_a^b \bm{F}(x(t), y(t)) \cdot
        \bm{T}(t) \dd{t},
      \end{align*}
      where $\bm{T}(t)$ is denotes the tangent vector at point $t$, and $x(a),
      y(a)$ and $x(b), y(b)$ are the end points of the curve $C$.
      We know that the tangent vector to $C$ is perpendicular to the vector
      field at every point. Thus the integrand vanishes and we get $\int_a^b 0
      \dd{t} = 0$. This completes the proof.
    \end{proof}
    \item \begin{proof}
        We are going to disprove it by a counter example. Let $\bm{F}(x,y) =
        \mqty[0, 1]$ be the vector field, and a parameterized curve $C$ a circle,
        which starts and ends at $(1,0)$ with radius $1$, which is
        $x(t)=\cos(t), y(t)=\sin(t), 0\le t\le 2\pi$. The tangent vector at
        point $(1,0)$ is $\mqty[0, 1]$, which is not perpendicular with
        $\bm{F}(0, 1) = \mqty[0, 1]$. Let $f(x,y)=y$, we then have $\grad f =
        \mqty[0, 1]$. By the fundamental theorem of calculus for line
        integrals, we have
        \begin{align*}
          \int_C \bm{F}\cdot \dd{\bm{r}}= f(0,0) - f(0,0) = 0.
        \end{align*}
        This completes the proof.
    \end{proof}
  \end{enumerate}
\end{Answer}

\begin{Answer}[number=36]
  \begin{enumerate}
    \item A piecewise smooth curve is a curve that can be write as a union of
      finite number of smooth curves.
    \item \begin{proof}
      We are going to prove this by using induction on the variable $n$. Let
      $C$ be a piecewise smooth curve that is $C = \sum_{i=1}^n C_i$ that
      starts at point $a$ and ends at point $b$, where $n$
      is an integer that is greater or equal to $1$. We are trying to show that
      \begin{align*}
        \int_C \grad f \cdot \dd{\bm{r}} = f(b) - f(a).
      \end{align*}
      \emph{Base case:}
      When $n=1$, we have $C = C_1$ and
      \begin{align*}
        \int_{C_1} \grad f \cdot \dd{\bm{r}} = f(b) - f(a)
      \end{align*}
      is true by the fundamental theorem of line integrals for smooth curves.

      \emph{Inductive step:}
      Assume that the proposition holds for some fixed integer $n\ge 1$, and we
      are trying to show that it is also true for $n+1$. Let point $c$ be the
      end point for the curve $\hat{C} = \sum_{i=1}^n C_i$. We then have
      \begin{align*}
        \int_{C}\grad f \cdot \dd{\bm{r}} &= \int_{\sum_{i=1}^{n+1} C_i} \grad
        f \cdot \dd{\bm{r}} \\
        &= \int_{\hat{C} + C_{n+1}} \grad f \cdot \dd{\bm{r}} \\
        &= [f(c) - f(a)] + \int_{C_{n+1}} \grad f \cdot \dd{\bm{r}} \\
        &= [f(c) - f(a)] + [f(b) - f(c)] \\
        &= f(b) - f(a).
      \end{align*}
      This completes the proof.
    \end{proof}
  \end{enumerate}
\end{Answer}

\begin{Answer}[number=37]
  \begin{enumerate}
    \item The parameterized function $-C$ will be $x=f\circ c (t)$ and $y =
      g\circ c (t)$, where $c(t) = -t$. Let $C$ be the curve $x=\cos(t),
      y=\sin(t), 0\le t\le \pi$, which starts at $(1,0)$ and ends at $(-1,0)$.
      Then $-C$ is $x=\cos(-t), y=\sin(-t), -\pi \le t\le -0$, which starts at
      $(-1,0)$ and ends at $(1,0)$.

    \item \begin{proof}
        We are going to prove this proposition by direct proof. Let $\bm{T}(t)$
        be the tangent vector of the curve $C$ at $t$, and $c(t) = -t$. Thus
        $\bm{T}(c(t))$ is the tangent vector of the curve $-C$ at $t$. We have
        \begin{align*}
          \int_{-C} \bm{F}\cdot \dd{\bm{r}} &= \int_{-b}^{-a} \bm{F}(f(c(t)),
          g(c(t))) \cdot \bm{T}(c(t)) \dd{t} \\
          &= \int_{b}^{a} \bm{F}(f(t), g(t)) \cdot \bm{T}(t) \dd{t} \\
          &= -\int_{b}^{a} \bm{F}(f(t), g(t)) \cdot \bm{T}(t) \dd{t} \\
          &= -\int_C \bm{F}\cdot \dd{\bm{r}}.
        \end{align*}
        This completes the proof.
    \end{proof}
  \end{enumerate}
\end{Answer}

\end{document}
