\documentclass[10pt,twocolumn]{article}
%\usepackage{amsmath,amssymb,amsthm,amsbsy,amsfonts,mathtools}
\usepackage{amsmath}
\usepackage{amssymb}
\usepackage{amsthm}
\usepackage{physics}
\usepackage{hyperref}
\usepackage{exercise}
\usepackage[makeroom]{cancel}
\usepackage[margin=2em]{geometry}

\usepackage{graphicx}

\usepackage{footmisc}
\DefineFNsymbols{mySymbols}{{\ensuremath\dagger}{\ensuremath\ddagger}\S\P
   *{**}{\ensuremath{\dagger\dagger}}{\ensuremath{\ddagger\ddagger}}}
\setfnsymbol{mySymbols}

\newcommand{\N}{\mathbb{N}}
\newcommand{\R}{\mathbb{R}}
\newcommand{\Z}{\mathbb{Z}}
\newcommand{\Q}{\mathbb{Q}}

\newenvironment{theorem}[2][Theorem]{\begin{trivlist}
\item[\hskip \labelsep {\bfseries #1}\hskip \labelsep {\bfseries #2.}]}{\end{trivlist}}
\newenvironment{lemma}[2][Lemma]{\begin{trivlist}
\item[\hskip \labelsep {\bfseries #1}\hskip \labelsep {\bfseries #2.}]}{\end{trivlist}}
\newenvironment{exercise}[2][Exercise]{\begin{trivlist}
\item[\hskip \labelsep {\bfseries #1}\hskip \labelsep {\bfseries #2.}]}{\end{trivlist}}
\newenvironment{problem}[2][Problem]{\begin{trivlist}
\item[\hskip \labelsep {\bfseries #1}\hskip \labelsep {\bfseries #2.}]}{\end{trivlist}}
\newenvironment{question}[2][Question]{\begin{trivlist}
\item[\hskip \labelsep {\bfseries #1}\hskip \labelsep {\bfseries #2.}]}{\end{trivlist}}
\newenvironment{corollary}[2][Corollary]{\begin{trivlist}
\item[\hskip \labelsep {\bfseries #1}\hskip \labelsep {\bfseries #2.}]}{\end{trivlist}}
\newenvironment{answer}[2][Answer]{\begin{trivlist}
\item[\hskip \labelsep {\bfseries #1}\hskip \labelsep {\bfseries #2.}]}{\end{trivlist}}

\let\emph\relax % there's no \RedeclareTextFontCommand
\DeclareTextFontCommand{\emph}{\bfseries\em}

\setlength{\columnseprule}{0.4pt}
\setlength{\columnsep}{3em}

\usepackage{todonotes}

\author{Yingbo Ma \\ Student ID: \tt{16058474}}
\date{April 11, 2018}
\title{Homework 1}

\begin{document}
\maketitle

\paragraph{Notation:}
I am going to use the expression $\lim_{m,n\to\infty}$ to denote
$\lim_{m\to\infty}\lim_{n\to\infty}$ throughout this homework.

\begin{exercise}{1}
  Let $R$ denote the rectangle $[0, 1]\times [0, 1]$. Using the limit definition,
  prove that $\iint_{R}e^{xy}\dd{A} \le 3$.
\end{exercise}

\begin{proof}
  We can expand the integral $\iint_{R}e^{xy}\dd{A}$ out by using the limit
  definition. We have
  \begin{align*}
    \iint_{R}e^{xy}\dd{A} &= \lim_{m,n\to \infty} \sum_{i=1}^m \sum_{j=1}^n
                            e^{x_{ij}^* \cdot y_{ij}^*} \Delta A \\
    &= \lim_{m,n\to \infty} \sum_{i=1}^m \sum_{j=1}^n
                            e^{x_{ij}^* \cdot y_{ij}^*} \cdot
                            (\frac{1}{m}\cdot \frac{1}{n}) \\
    &= \lim_{m,n\to \infty} \sum_{i=1}^m \sum_{j=1}^n
                            e^{x_{ij}^* \cdot y_{ij}^*} \cdot
                            \frac{1}{mn}.
  \end{align*}
  We know that $e^{xy}$ has a gradient greater than $0$ in the domain
  $[0,1]\times[0,1]$, and $x\cdot y$ is in $[0,1]$ for $x$ in $[0,1]$ and $y$
  in $[0,1]$. In addition, $e^1$ is less than or equal to $3$. Hence, the range
  of $e^{x\cdot y}$ is less than or equal to $3$, for $x$ in $[0,1]$ and $y$ in
  $[0,1]$.
  \begin{align*}
    \lim_{m,n\to \infty} \sum_{i=1}^m \sum_{j=1}^n
                            e^{x_{ij}^* \cdot y_{ij}^*} \cdot
                            \frac{1}{mn}
    &\le \lim_{m,n\to \infty} \sum_{i=1}^m \sum_{j=1}^n
                            3 \cdot
                            \frac{1}{mn} \\
    &= \lim_{m,n\to \infty} 3\sum_{i=1}^m \sum_{j=1}^n
                              \frac{1}{mn} \\
    &= \lim_{m,n\to \infty} 3\cdot \frac{1}{mn}\cdot m\cdot n = 3.
  \end{align*}
  Thus, we have $\iint_{R}e^{xy}\dd{A} \le 3$ where $R$ is the rectangle
  $[0,1]\times [0,1]$. This completes the proof.
\end{proof}

\begin{exercise}{2}
  Prove or disprove: if $R$ is a rectangle located entirely within the third quadrant, i.e.,
  in the region where $x < 0$ and $y < 0$, then $\iint_R f(x, y) dA \le 0$.
\end{exercise}

\begin{proof}
  The previous statement is false. If we choose the function $f(x,y)$ to be
  $f(x,y) = 1$, and $R$ to be $[-3,-2]\times [-3,-2]$, then use the limit
  definition the expand the integral $\iint_R f(x, y) dA \le 0$. We have
  \begin{align*}
    \iint_{R}f(x,y)\dd{A} &= \lim_{m,n\to \infty} \sum_{i=1}^m \sum_{j=1}^n
                            f(x_{ij}^*, y_{ij}^*) \Delta A \\
    &=\lim_{m,n\to \infty} \sum_{i=1}^m \sum_{j=1}^n
                            1 \cdot (\frac{1}{m}\cdot \frac{1}{n}) \\
    &=\lim_{m,n\to \infty} 1\cdot\sum_{i=1}^m \sum_{j=1}^n
                            (\frac{1}{mn}) = 1 > 0.
  \end{align*}
  Thus, the previous statement is false. This completes the proof.
\end{proof}

\begin{exercise}{3}
  Prove that if $n\ge 1$ is an integer, then $\frac{n^3}{3} + \frac{n^2}{2} +
  \frac{n}{6}$ is an integer.
\end{exercise}

\begin{proof}
  I am going to prove this by using induction on the variable $n$.

  \emph{Base case:} If $n=1$, then $\frac{1^3}{3} + \frac{1^2}{2} + \frac{1}{6}
  = 1$, which is an integer, hence base case holds.

  \emph{Induction step:} Fix any integer $k\ge 1$, and assume $\frac{k^3}{3} +
  \frac{k^2}{2} + \frac{k}{6}$ is an integer, then we want to show
  $\frac{(k+1)^3}{3} + \frac{(k+1)^2}{2} + \frac{k+1}{6}$ is an integer. We
  compute
  \begin{align*}
    &\frac{(k+1)^3}{3} + \frac{(k+1)^2}{2} + \frac{k+1}{6} \\
    =&\frac{1}{6}\cdot(2k^3+9k^2+13k+6) \\
    =&\frac{1}{6}\cdot(2k^3+3k^2+k) + \frac{1}{6}\cdot(6k^2+12k+6) \\
    =&\frac{1}{6}\cdot(2k^3+3k^2+k) + (k^2+2k+1) \\
    =&\underbrace{\underbrace{\frac{k^3}{3} + \frac{k^2}{2} + \frac{k}{6}}_\text{is an
    integer by induction hypothesis} + \underbrace{(k+1)^2}_\text{is an
    integer}}_\text{is an integer}.
  \end{align*}
  Thus, the induction step holds. Hence we have proved the proposition by
  induction for all $n\ge 1$.
\end{proof}

\begin{exercise}{4}
  Let $f(x,y)=x^2-y^2$. Let $R$ denote the rectangle $-3\le x \le 3$ and $-1\le
  y \le 1$. Divide the $x$-interval into three equal segments and keep the
  $y$-interval as one segment. What sample point can you choose the make the
  Riemann sum for $f(x,y)$ as big as possible? Compute that Riemann sum. What
  sample points can you make could you choose to make the Riemann sum as small
  as possible? Compute that Riemann sum. You got two different values, does
  that mean the double integral does not exit?
\end{exercise}

\begin{answer}{4}
  I can choose the simple points of $(-3, 0)$, $(-1, 0)$ and $(3, 0)$ to make
  the Riemann sum as big as possible. We have
  \begin{align*}
    &\iint_{R}f(x,y)\dd{A} = \sum_{i=1}^3 \sum_{j=1}^1
                            f(x_{ij}^*, y_{ij}^*) \Delta A \\
    =&\sum_{i=1}^3 \sum_{j=1}^1 f(x_{ij}^*, y_{ij}^*) \cdot
    (\frac{3-(-3)}{3}\cdot \frac{1-(-1)}{1}) \\
    =&(((-3)^2 - 0^2) + ((-1)^2 - 0^2) + (3^2 - 0^2)) \cdot 4 = 76
  \end{align*}
  I can choose the simple points of $(-1, 1)$, $(0, 1)$ and $(1, 1)$ to make
  the Riemann sum as small as possible. The Riemann sum is
  \begin{align*}
    &\iint_{R}f(x,y)\dd{A} = \sum_{i=1}^3 \sum_{j=1}^1
                            f(x_{ij}^*, y_{ij}^*) \Delta A \\
    =&(((-1)^2 - 1^2) + (0^2 - (-1)^2) + (1^2 - 1^2)) \cdot 4 = -4
  \end{align*}
  The integral still exists, because the theorem states that when the limit of
  $m$ and $n$ tend to infinity, however, in this case, the $m$ and $n$ are
  finite numbers.
\end{answer}

\begin{exercise}{5}
  Prove that the upper sum for the Dirichlet function
  \[
    f(x) = \begin{cases}
      0& \qq{if $x$ is rational}\\
      1& \qq{if $x$ is irrational}
    \end{cases}
  \]
  over the interval $[0,5]$ is equal to 5.
\end{exercise}

\begin{proof}
  We can expand the integral by its limit definition.
  \begin{align*}
    \int_0^5f(x)\dd{x} &= \lim_{m\to \infty} \sum_{i=1}^m f(x_{i}^*) \Delta x \\
    &= \lim_{m\to \infty} \sum_{i=1}^m f(x_{i}^*) \cdot \frac{5}{m}
  \end{align*}
  We can choose any point in $[\frac{5(i-1)}{m},\frac{5i}{m}]$ for the point
  $x_i^*$. The upper sum of an integral is the largest possible value of the
  sum, thus we have
  \begin{align*}
    \int_0^5f(x)\dd{x} &= \lim_{m\to \infty} \sum_{i=1}^m 5 \cdot \frac{5}{m} =
    5,
  \end{align*}
  since between any two distinct rational numbers exists some irrational
  number. Hence, the proof is complete.
\end{proof}

\begin{exercise}{6}
  By altering the function from exercise 5, give an example of a function
  $f(x,y)$ for which the double integral of $f(x,y)$ over the rectangle
  $[0,1]\times[0,1]$ does not exist. You don't have to prove your answer is
  correct.
\end{exercise}

\begin{answer}{6}
  The double integral of the function
  \[
    f(x,y) = \begin{cases}
      0& \qq{if $x+y$ is rational}\\
      1& \qq{if $x+y$ is irrational}
    \end{cases}
  \]
  over the rectangle $[0,1]\times[0,1]$ does not exist.
\end{answer}

\begin{exercise}{7}
  If $x$ is rational and $y$ is irrational, then $x + y$ is what? (Is it always
  rational, always irrational, or do we not have enough information to say?)
  Prove your answer using proof by contradiction.
\end{exercise}

\begin{proof}
  Let us assume that $x+y$ is rational. By the question, $x$ is a rational
  number and $y$ is an irrational number. We can represent $x$ by
  $\frac{a}{b}$, and represent $x+y$ by $\frac{c}{d}$, where $a,b,c$ and $d$
  are all integers. Thus, we have
  \begin{align*}
    x + y &= \frac{a}{b} + y \\
    \frac{c}{d} &= \frac{a}{b} + y \\
    y &= \frac{c}{d} - \frac{a}{b} = \frac{bc-ad}{bd}.
  \end{align*}
  Hence, we can conclude that $y$ is a rational number, which contradicts the
  assumption. Hence, $x+y$ must be an irrational number. This completes the
  proof.
\end{proof}

\begin{exercise}{8}
  Proof that if $n\in \mathbb{N}$, then
  \[
    \frac{1}{2!} + \frac{2}{3!} + \cdots + \frac{n}{(n+1)!} = 1 -
    \frac{1}{(n+1)!}
  \]
\end{exercise}

\begin{proof}
  I am going to prove this by using induction on the variable $n$.

  \emph{Base case:} If $n=1$, then $\frac{1}{2!} = \frac{1}{2}$, and
  $1-\frac{1}{(n+1)!}  = 1-\frac{1}{2!} = \frac{1}{2}$. Therefore the base case
  holds.

  \emph{Induction step:} Fix any integer $k\ge 1$, and assume $\frac{1}{2!} +
  \frac{2}{3!} + \cdots + \frac{k}{(k+1)!} = 1 - \frac{1}{(k+1)!}$, then we
  want to show $\frac{1}{2!} + \frac{2}{3!} + \cdots + \frac{k+1}{(k+2)!} = 1 -
  \frac{1}{(k+2)!}$. We have
  \begin{align*}
    &\frac{1}{2!} + \frac{2}{3!} + \cdots + \frac{k+1}{(k+2)!} \\
    =&\frac{1}{2!} + \frac{2}{3!} + \cdots + \frac{k}{(k+1)!} +
    \frac{k+1}{(k+2)!} \\
    =&1 - \frac{1}{(k+1)!} + \frac{k+1}{(k+2)!} \\
    =&1 - \frac{k+2}{(k+2)!} + \frac{k+1}{(k+2)!} \\
    =&1 - \frac{1}{(k+2)!}.
  \end{align*}
  Thus, the induction step holds. Thus we have proved the proposition by
  induction for all $n\ge 1$.
\end{proof}

\begin{exercise}{9}
  Indentify the error in the following famous ``proof'' by induction. Claim:
  All horses are the same color.
\end{exercise}

\begin{answer}{9}
  The error occurs in the induction step, it claims ``the first horse has the
  same color as any middle horse.'' It assumes that for $k+1$ horses there is a
  non-empty intersection between the set of $k$ horses without the first one
  and the set of $k$ horses without the last one. However, this assumption is
  not true for the case of two horses. Thus, the induction step does not hold.
  Therefore, the ``proof'' is not correct.
\end{answer}

\[
  \frac{2}{9\pi}\int_{0}^{\pi} \int_{0}^{3} r \cdot \sin \theta r \dd{r}\dd{\theta}
\]

\[
  \frac{2}{9\pi}\int_{0}^{\sqrt{9-x^2}} \int_{-3}^{3} y \dd{x}\dd{y} =
  \frac{2}{9\pi}\int_0^\pi \frac{r^3}{3}\sin\theta \Big\vert_{r=0}^3 \dd{\theta}
  \dd{\theta} =
  \frac{2}{9\pi}\int_0^\pi \frac{r^3}{3}9\sin\theta \dd{\theta} = \frac{4}{\pi}
\]

Reminder: Integral is a limit of Riemann sums.
Idea: subdivide the region we are integrating over into smaller and samller
pieces.

Polar, pieces further from origin have bigger area, $\approx \Delta r \Delta
\theta$. ``Polar rectangle''.

\end{document}
