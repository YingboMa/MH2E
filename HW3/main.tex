\documentclass[8pt,twocolumn]{article}
%\usepackage{amsmath,amssymb,amsthm,amsbsy,amsfonts,mathtools}
\usepackage{amsmath}
\usepackage{amssymb}
\usepackage{amsthm}
\usepackage{physics}
\usepackage{hyperref}
\usepackage{exercise}
\usepackage[makeroom]{cancel}
\usepackage[margin=2em]{geometry}

\usepackage{graphicx}

\usepackage{footmisc}
\DefineFNsymbols{mySymbols}{{\ensuremath\dagger}{\ensuremath\ddagger}\S\P
   *{**}{\ensuremath{\dagger\dagger}}{\ensuremath{\ddagger\ddagger}}}
\setfnsymbol{mySymbols}

\newcommand{\N}{\mathbb{N}}
\newcommand{\R}{\mathbb{R}}
\newcommand{\Z}{\mathbb{Z}}
\newcommand{\Q}{\mathbb{Q}}

\newenvironment{theorem}[2][Theorem]{\begin{trivlist}
\item[\hskip \labelsep {\bfseries #1}\hskip \labelsep {\bfseries #2.}]}{\end{trivlist}}
\newenvironment{lemma}[2][Lemma]{\begin{trivlist}
\item[\hskip \labelsep {\bfseries #1}\hskip \labelsep {\bfseries #2.}]}{\end{trivlist}}
\newenvironment{exercise}[2][Exercise]{\begin{trivlist}
\item[\hskip \labelsep {\bfseries #1}\hskip \labelsep {\bfseries #2.}]}{\end{trivlist}}
\newenvironment{problem}[2][Problem]{\begin{trivlist}
\item[\hskip \labelsep {\bfseries #1}\hskip \labelsep {\bfseries #2.}]}{\end{trivlist}}
\newenvironment{question}[2][Question]{\begin{trivlist}
\item[\hskip \labelsep {\bfseries #1}\hskip \labelsep {\bfseries #2.}]}{\end{trivlist}}
\newenvironment{corollary}[2][Corollary]{\begin{trivlist}
\item[\hskip \labelsep {\bfseries #1}\hskip \labelsep {\bfseries #2.}]}{\end{trivlist}}
\newenvironment{answer}[2][Answer]{\begin{trivlist}
\item[\hskip \labelsep {\bfseries #1}\hskip \labelsep {\bfseries #2.}]}{\end{trivlist}}

\let\emph\relax % there's no \RedeclareTextFontCommand
\DeclareTextFontCommand{\emph}{\bfseries\em}

\setlength{\columnseprule}{0.4pt}
\setlength{\columnsep}{3em}

\usepackage{todonotes}

\author{Yingbo Ma \thanks{Student ID: \tt{16058474}}}
\title{\vspace{-1.cm}Homework 2}
\date{April 22, 2018}

\begin{document}
\maketitle

\begin{Answer}[number=17]
\begin{proof}
  I am going to prove this by using induction on the variable $n$.

  \emph{Base case 1:} If $n=1$, then $F_1 = 1 \ge \varphi^{-1} \approx 0.618$.
  Therefore the first base case holds.

  \emph{Base case 2:} If $n=2$, we have $F_2 = 1 \ge \varphi^{0} = 1$.
  Therefore the second base case holds.

  \emph{Induction step:} Fix any integer $k\ge 1$, we assume $F_k\ge
  \varphi^{k-2}$ and $F_{k-1}\ge\varphi^{k-3}$, then we want to show $F_{k+1}
  \ge \varphi^{k-1}$ holds. First
  we know that
  \begin{align*}
    \varphi^2 &= 1+\varphi \\
    \varphi^2\cdot\varphi^{k-3} &= (1+\varphi)\cdot\varphi^{k-3} \\
    \varphi^{k-1} &= \varphi^{k-3} + \varphi^{k-2}.
  \end{align*}
  We can then compute
  \begin{align*}
    F_{k+1} = F_k + F_{k-1} \ge \varphi^{k-2} + \varphi^{k-3} = \varphi^{k-1}
  \end{align*}
  Thus, the induction step holds. Thus we have proved the proposition by
  induction for all $n\ge 1$.
\end{proof}
\end{Answer}

\begin{Answer}[number=18]
  \begin{enumerate}
    \item The symbol $m$ represents the number of horses in the domain of
      integration.
    \item The expression $(x_k^*,y_k^*)$ represents any fixed point that is in
      the $k$-th horse.
    \item The expression $\Delta A$ represents the area of a single horse.
  \end{enumerate}
\end{Answer}

\begin{Answer}[number=19]
  \emph{15.6.19:}
  In this question, the domain of integration is bounded by $x=0, y=0$, and
  $2x+y+z=4$. Thus, we have $z=4-2x-y$, $y=4-2x$ and $x=0$, if we solve for $z$
  and set another variable to $0$.
  \[D = \{(x,y,z) \,\big\vert\, 0\ge x\ge 2, 0\ge y\ge 4-2x, 0\ge 4-2x-y\},\] thus the
  integral is
  \[\int_0^2\int_0^{4-2x}\int_0^{4-2x-y}\dd{z}\dd{y}\dd{x}\].
  \emph{15.7.20:}
  The domain of integration is bounded by $x^2+y^2=1, x^2+y^2=16, z=0$, and
  $z=y+4$. We can convert them into a cylindrical coordinate. We have $r=1,
  r=4, z=0$, and $z=r\sin(\theta)+4$. Also, we know that $x$ and $y$ in a
  cylindrical coordinate are $r\cos\theta$ and $r\sin\theta$ respectively. We
  are going to integrate in the set
  \[
    D = \{(\theta, r, z) \,\big\vert\, 0\ge\theta\ge2\pi, 0\ge r\ge4, 0\ge z\ge
    r\sin(\theta)+4\}.
  \]
  Thus, the integral is
  \begin{align*}
    \int_0^4\int_0^{2\pi}\int_0^{r\sin(\theta)+4} (r\cos\theta-
    r\sin\theta)\cdot r\dd{z}\dd{\theta}\dd{r}
  \end{align*}
  \emph{15.7.23:}
  \begin{align*}
    &D = \{(r,\theta,z) \,\big\vert\, 0\le \theta\le2\pi, 0\le r\le 1, r\le z\le
    \sqrt{2-r^2}\} \\
    &\int_0^1\int_0^{2\pi}\int_r^{2-r^2} r\dd{z}\dd{\theta}\dd{r}
  \end{align*}
  \emph{15.8.24:}
  \begin{align*}
    &D = \{(\rho, \theta, \phi) \,\big\vert\, 0\le \rho \le 3, 0\le \theta\le \pi,
    0\le\phi\le \pi \} \\
    &\int_0^{2\pi}\int_0^{\frac{\pi}{2}}\int_0^3
    \left[\rho\sin(\theta)\sin(\phi)\right]^2\cdot\rho^2\cdot\sin(\theta) \dd{\rho}\dd{\theta}\dd{\phi}
  \end{align*}
  \emph{15.8.26:}
  \begin{align*}
    &D = {(\rho, \theta, \phi)\,\big\vert\,1\le\rho\le2, 0\le\theta\le2\pi,
    0\le\phi\le\frac{\pi}{4}} \\
    &\int_0^{\frac{\pi}{4}}\int_0^{2\pi}\int_1^2 \rho \cdot \rho^2\sin(\phi) \dd{\rho}\dd{\theta}\dd{\phi}
  \end{align*}
\end{Answer}

\begin{Answer}[number=20]
  \emph{15.6.31}
  \begin{align*}
    E =&  \{ (x,y,z) \, \big\vert \, -2 \le x \le 2, x^2 \le y \le 4,  0 \le z \le 2-\frac{y}{2} \} \\
    = &\{ (x,y,z) \, \big\vert \, -\sqrt{y} \le x\le \sqrt{y}, 0 \le y \le 4, 0 \le z \le 2-\frac{y}{2} \} \\
    = &\{ (x,y,z) \, \big\vert \, -\sqrt{y} \le x\le \sqrt{y}, 0 \le y \le 4, 0 \le z \le 2-\frac{y}{2} \} \\
    = &\{ (x,y,z) \, \big\vert \, -\sqrt{y} \le x\le \sqrt{y}, 0 \le y \le 4-2z, 0 \le z \le 2 \} \\
    = &\{ (x,y,z) \, \big\vert \, -2 \le x \le 2, x^2 \le y \le 4-2z, 0 \le z \le 2-\frac{x^2}{2} \} \\
    = &\{ (x,y,z) \, \big\vert \, -\sqrt{4-2z} \le x\le \sqrt{4-2z},\\ &x^2 \le y \le
    4-2z, 0 \le z \le 2 \}.
  \end{align*}
  Thus, the integrals are
  \begin{align*}
    &\int_{-2}^2 \int_{x^2}^4
    \int_0^{2-\frac{y}{2}} f(x,y,z) \dd{z}\dd{y}\dd{x} \\
 = &\int_{-\sqrt{y}}^{\sqrt{y}} \int_0^4 \int_0^{2-\frac{y}{2}} f(x,y,z)\dd{z}\dd{y}\dd{x} \\
 = &\int_{-\sqrt{y}}^{\sqrt{y}} \int_0^4 \int_0^{2-\frac{y}{2}} f(x,y,z)\dd{z}\dd{y}\dd{x} \\
 = &\int_{-\sqrt{y}}^{\sqrt{y}} \int_0^{4-2z} \int_0^2 f(x,y,z) \dd{z}\dd{y}\dd{x} \\
 = &\int_{-2}^2 \int_{x^2}^{4-2z} \int_0^{2-\frac{x^2}{2}} f(x,y,z) \dd{z}\dd{y}\dd{x} \\
 = &\int_{-\sqrt{4-2z}}^{\sqrt{4-2z}} \int_{x^2}^{4-2z} \int_0^2 f(x,y,z) \dd{z}\dd{y}\dd{x}.
  \end{align*}
\end{Answer}

\end{document}
