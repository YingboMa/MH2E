\documentclass[8pt]{article}
%\usepackage{amsmath,amssymb,amsthm,amsbsy,amsfonts,mathtools}
\usepackage{amsmath}
\usepackage{amssymb}
\usepackage{amsthm}
\usepackage{physics}
\usepackage{hyperref}
\usepackage{exercise}
\usepackage[makeroom]{cancel}
\usepackage[margin=2em]{geometry}

\usepackage{graphicx}

\usepackage{footmisc}
\DefineFNsymbols{mySymbols}{{\ensuremath\dagger}{\ensuremath\ddagger}\S\P
   *{**}{\ensuremath{\dagger\dagger}}{\ensuremath{\ddagger\ddagger}}}
\setfnsymbol{mySymbols}

\newcommand{\N}{\mathbb{N}}
\newcommand{\R}{\mathbb{R}}
\newcommand{\Z}{\mathbb{Z}}
\newcommand{\Q}{\mathbb{Q}}

\newenvironment{theorem}[2][Theorem]{\begin{trivlist}
\item[\hskip \labelsep {\bfseries #1}\hskip \labelsep {\bfseries #2.}]}{\end{trivlist}}
\newenvironment{lemma}[2][Lemma]{\begin{trivlist}
\item[\hskip \labelsep {\bfseries #1}\hskip \labelsep {\bfseries #2.}]}{\end{trivlist}}
\newenvironment{exercise}[2][Exercise]{\begin{trivlist}
\item[\hskip \labelsep {\bfseries #1}\hskip \labelsep {\bfseries #2.}]}{\end{trivlist}}
\newenvironment{problem}[2][Problem]{\begin{trivlist}
\item[\hskip \labelsep {\bfseries #1}\hskip \labelsep {\bfseries #2.}]}{\end{trivlist}}
\newenvironment{question}[2][Question]{\begin{trivlist}
\item[\hskip \labelsep {\bfseries #1}\hskip \labelsep {\bfseries #2.}]}{\end{trivlist}}
\newenvironment{corollary}[2][Corollary]{\begin{trivlist}
\item[\hskip \labelsep {\bfseries #1}\hskip \labelsep {\bfseries #2.}]}{\end{trivlist}}
\newenvironment{answer}[2][Answer]{\begin{trivlist}
\item[\hskip \labelsep {\bfseries #1}\hskip \labelsep {\bfseries #2.}]}{\end{trivlist}}

\let\emph\relax % there's no \RedeclareTextFontCommand
\DeclareTextFontCommand{\emph}{\bfseries\em}

\setlength{\columnseprule}{0.4pt}
\setlength{\columnsep}{3em}

\usepackage{todonotes}

\author{Yingbo Ma \thanks{Student ID: \tt{16058474}}}
\title{\vspace{-1.cm}Homework 4}
\date{May 07, 2018}

\begin{document}
\maketitle

\begin{Answer}[number=21]
  The transformation $T$ that maps $R$ to $S$ is formed by functions
  \(
    \begin{cases}
      u(x,y) = x+y \\
      v(x,y) = x-y
    \end{cases},
  \)
  and $S$ is then bounded by $u=0, u=3, v=0,$ and $v=2$. Thus the map from $S$
  to $R$ is
  \(
    \begin{cases}
      x(u,v) = \frac{u+v}{2} \\
      y(u,v) = \frac{u-v}{2}
    \end{cases}.
  \)
  Then the Jacobian matrix is
  \[
    J(x,y)={\begin{bmatrix}
      \pdv{x}{u}&\pdv{x}{v}\\
      \pdv{y}{u}&\pdv{y}{v}
    \end{bmatrix}}=
    \begin{bmatrix}1/2&1/2\\1/2&-1/2\end{bmatrix},
  \]
  and $\abs{\det J} = \frac{1}{2}$. Thus, the integral is now
      \[
    \iint_R (x+y)e^{x^2-y^2}\dd{A} = \int_0^3\int_0^2
    ue^{uv}\frac{1}{2}\dd{v}\dd{u}.
  \]
\end{Answer}

\begin{Answer}[number=22]
  Let $T_2:(x,y)\mapsto (u,v)$ be given by $(x,y)\mapsto(x+y, x-y)$.
  Let $T_1:(u,v)\mapsto (x,y)$ be given by $(u,v)\mapsto((u+v)/2, (u-v)/2)$.
  The region $R$ consists of all the points in the $xy$-plane satisfying $y\le
  -x+1, y\ge x-1, y\ge -x-1,$ and $y\le x+1$.
  The region $S$ consists of all the points in the $uv$-plane satisfying $u\le
  1, v\le 1, u\ge -1,$ and $v\ge -1$.
  We can proof that $T_1$ maps $S$ to $R$ by direct proof. We then have
  \begin{align*}
    y\le -x+1 & \Rightarrow y+x \le 1  \Rightarrow u \le 1   \\
    y\ge x-1  & \Rightarrow y-x \ge -1 \Rightarrow v \le 1   \\
    y\ge -x-1  & \Rightarrow y+x \ge -1 \Rightarrow u \ge -1  \\
    y\le x+1  & \Rightarrow y-x \le 1  \Rightarrow v \ge -1.
  \end{align*}
  We can also proof that $T_2$ maps $R$ to $S$ by direct proof. We have
  \begin{align*}
    u \le 1  & \Rightarrow 2u \le 2  \Rightarrow \frac{u+v}{2} + \frac{u-v}{2} \le 1
    \Rightarrow x+y \le 1 \Rightarrow y\le -x+1\\
    u \ge -1 & \Rightarrow 2u \ge -2 \Rightarrow \frac{u+v}{2} + \frac{u-v}{2} \ge -1
    \Rightarrow x+y \ge -1 \Rightarrow y\ge x-1\\
    v \le 1  & \Rightarrow 2v \le 2  \Rightarrow \frac{v+u}{2} + \frac{v-u}{2} \le 1
    \Rightarrow x-y \le 1 \Rightarrow y\ge x-1\\
    v \ge -1 & \Rightarrow 2v \ge -2 \Rightarrow \frac{v+u}{2} + \frac{v-u}{2} \ge -1
    \Rightarrow x-y \ge -1 \Rightarrow y\le x+1.
  \end{align*}
  Thus, we have demonstrated that $T2: R\mapsto S$ and $T1: S\mapsto R$. We
  then show that $T_1 \circ T_2$ is the identity map on $R$:
  \begin{align*}
    &T_1(T_2(x,y)) = T_1(x+y, x-y) \\= &(\frac{x+y+x-y}{2}, \frac{x+y-x+y}{2}) =
    (x,y).
  \end{align*}
  We next show that $T_2 \circ T_1$ is the identity map on $S$:
  \begin{align*}
    &T_2(T_1(u,v)) = T_2(\frac{u+v}{2}, \frac{u-v}{2}) \\ =
    &(\frac{u+v}{2}+\frac{u-v}{2}, \frac{u+v}{2}-\frac{u-v}{2})=
    (u,v).
  \end{align*}
  The Jacobian matrix is
  \[
    J(x,y)={\begin{bmatrix}
      \pdv{x}{u}&\pdv{x}{v}\\
      \pdv{y}{u}&\pdv{y}{v}
    \end{bmatrix}}=
    \begin{bmatrix}1/2&1/2\\1/2&-1/2\end{bmatrix},
  \]
  since we have the transformation
  \(
    \begin{cases}
      u(x,y) = x+y \\
      v(x,y) = x-y
    \end{cases},
  \)
  and $\abs{\det J} = \frac{1}{2}$.
  Thus, we have the integral as
  \[
    \iint_R e^{x+y} \dd{A} = \int_0^1\int_0^1\frac{1}{2} e^u\dd{u}\dd{v} =
    \frac{1}{2}\eval{e^u}_{-1}^1 \eval{v}_{-1}^1 = e - \frac{1}{e}
  \]
\end{Answer}

\begin{Answer}[number=23]
  \begin{lemma}\label{lm:matrix}
    The determinant of a $n$ by $n$ diagonal matrix $A_{n,n}$ which is filled with
    real number $r$ in the diagonal is $r^n$ for any integer $n\ge 1$.
  \end{lemma}
  \begin{proof}
    We are going to prove the above lemma by using induction on the variable
    $n$. The matrix $A_{n,n}$ is a $n$ by $n$ matrix that is in form of
    $\mqty(\dmat{r,\ddots,r})$, where $r$ is a real number.

    \emph{Base case:} When $n=1$, we have $\det(A_{1,1}) = \det(r) = r = r^1$. Thus
    the base case holds.

    \emph{Inductive step:} For some fixed integer $n\ge 1$, and assume
    $\det(A_{n,n}) = r^n$, then we want to show $\det(A_{n+1,n+1}) = r^{n+1}$.
    By using the Laplace expansion on the first row, we have
    \begin{align*}
      \det(A_{n+1,n+1}) = \sum_{j=1}^{N+1}(-1)^{1+j}a_{1j}\det(M_{1j}).
    \end{align*}
    By the sparsity pattern of the matrix $A_{n+1,n+1}$, we know the above
    expression evaluates to
    \begin{align*}
      1\cdot r\cdot \det(A_{n,n}) = r\cdot r^n = r^{n+1}.
    \end{align*}
    This completes the proof.
  \end{proof}
  Here is the proof for exercise 23.
  \begin{proof}
    The transformation $T$ that maps the unit ball in $\R^n$ to the ball of
    radius $r$ in $R^n$ is given by
    \(
      T(x) = A_{n,n} x
    \)
    where $A_{n,n}$ is an $n$ by $n$ diagonal matrix, and it is filled with
    real number $r$ in the diagonal. From \cref{lm:matrix}, we know that its
    determinant is $r^n$, this implies that for an area enclosed, under the
    transform, its area is now $r^n$ times. When $r$ is not zero, then, the
    transformation is one-to-one and onto, since the linear transformation
    matrix has a non-zero determinant. When $r$ is zero, the transformation is
    not one-to-one nor onto, however, such transformation maps a
    $n$-dimensional unit ball into a point which has zero volume, and the point
    is the boundary itself, so we can still use Jacobian to perform the change
    of variables. We have $\det(J) = \det(A_{n,n}) = r^n$ for integer $n\ge 1$.
    We then have
    \begin{align*}
      V_1 &= \int_{\sum_{i=1}^n x^i\le 1} 1 \dd{x} \\
      V_2 &= \int_{\sum_{i=1}^n x^i\le r^n} 1 \dd{u} = \int_{\sum_{i=1}^n x^i\le 1} 1 \abs{\det(J)} \dd{x} = \abs{\det(J)}
      V_1.
    \end{align*}
    Where $V_1$ represents the volume for the unit ball, and $V_2$ represents
    the volume for the scaled ball. This completes the proof.
  \end{proof}
\end{Answer}

\begin{Answer}[number=24]
  We can transform the equality $ax^2+ay^2=cz^2, z\ge 0$ to
  \begin{align*}
    x^2+y^2&=\frac{c}{a}z^2 \\
    \rho^2 \sin^2\varphi&=\frac{c}{a}\rho^2\cos^2\varphi \\
    \tan^2\varphi&=\frac{c}{a} \\
    \varphi&=\tan^{-1}\sqrt{\frac{c}{a}}.
  \end{align*}
  Thus, it is a cone which has the angle of $\tan^{-1}\sqrt{\frac{c}{a}}$ away
  from the $z$ axis.
\end{Answer}

\begin{Answer}[number=25]
  \begin{enumerate}
    \item \begin{proof}
      If $g\circ f$ is one-to one, $f$ is one-to-one must be. Assume towards
      contradiction that $f$ is not one-to-one. That implies that that for some
      value $a\ne b$, $f(a)=f(b)$ holds. In addition, we have $g(f(a)) =
      g(f(b))$, yet $a\ne b$. However, $g\circ f$ is one-to-one. Hence a
      contradiction. This completes the proof.
    \end{proof}
    \item \begin{proof}
        If $g\circ f$ is one-to one, $g$ does not have to be one-to-one. A
        counter example of the proposition can be given as
        \(f(x) = \begin{bmatrix}
          \imat{2} \\ 0 & 0
        \end{bmatrix}x\), which is one-to-one, since the number of
        pivots and the number of columns are the same, and \(g(x) = \begin{bmatrix}
          \mqty{\imat{2}} & \mqty{0\\0}
        \end{bmatrix}x\) which is not one-to-one, since the number of pivots
        and the number of columns are different. Thus, we have $(f\circ g) (x)
        = \begin{pmatrix}
          \imat{2}
        \end{pmatrix}x$, which is one-to-one, since the number of
        pivots and the number of columns are the same. This completes the
        proof.
    \end{proof}
  \end{enumerate}
\end{Answer}

\begin{Answer}[number=26]
  \begin{enumerate}
    \item A example can be $f(x)=10^{-10}\sin(10^{20}x)$. We have
      $\abs{f(x)}=10^{-10}\le.01$ for all $x$ and $\abs{f'(x)}=10^{10}\ge10^6$ for some $x$.
    \item The plot for the function $f(x)=10^{-10}\sin(10^{20}x)$ in the
      $xy$-plane is

    \begin{tikzpicture}
      \begin{axis}[width=4in,%north east,%axis equal image,
        domain=0:7*10^(-20),
        axis lines=middle,
        enlargelimits,
        axis line style={shorten >=-0.25cm,shorten <=-0.25cm,latex-latex},
        ticklabel style={fill=white},
        extra x ticks={0},
        xlabel=$x$,ylabel=$y$,
        clip=false,]
        \addplot[samples=200]{10^(-10)*sin(10^(20)*deg(x))};% node[fill=white, above]{$10^{-10}\sin(10^{20}x)$};
      \end{axis}
    \end{tikzpicture}
    \item \begin{proof}
        We can pick a function that is in the form of $f(x) = a\sin(bx)$ and
        $f'(x)=ab\sin(bx)$. For every fixed $\varepsilon> 0$ and $N> 0$, to
        fulfill $\abs{f(x)}<\varepsilon$ for all $x$, $\abs{a} < \varepsilon$
        must holds. To fulfill $f'(x)\ge N$ for $x$ that makes $f'(x)=\abs{a}b$,
        we have
        \begin{align*}
          \abs{a}b &\ge N, \\
          N/b &\le \abs{a} < \varepsilon  \\
          b/N &> 1/\varepsilon  \\
          b &> N/\varepsilon.
        \end{align*}
        This completes the requirements.
    \end{proof}
  \end{enumerate}
\end{Answer}

\begin{Answer}[number=27]
  This picture can be used to proof that \(\det(\mqty[c & a \\ d & b]) =
  cb-ad\). A square with the
  vertexes of $(0,0), (1, 0), (1, 0)$ and $(1,1)$, which has the area of $1$,
  is mapped to a parallelogram with vertexes $(0,0), (c,d), (a, b)$ and $(c+a,
  d+b)$ by the above matrix, which has the area of $(c+a)(b+d)-2ad-cd-ab=cb+cd+ab+ad-2ad-cd-ab =
  cb-ad$ from the diagram, which should equal to the matrix's determinant.
\end{Answer}

\begin{Answer}[number=28]
  \emph{16.2.4:} The curve can be parametrized by \(\begin{cases}
    x(t) &= 3t+2\\
    y(t) &= 4t
  \end{cases}\), for $0\le t\le 1$. We have $x'(t) = 3$ and $y'(t) = 4$.
  We then have the integral as
  \[
    \int_C xe^y\dd{s} = \int_0^1 (3t+2)e^{4t}\sqrt{3^2+4^2}\dd{t}.
  \]

  \emph{16.2.6:} We have the parametric equations as \(\begin{cases}
    x(t) &= t^3\\
    y(t) &= t
  \end{cases}\), for $-1\le t\le 1$. We have $x'(t) = 3t^2$.
  We then have the integral as
  \[
    \int_C e^x\dd{x} = \int_{-1}^1 e^{t^3} \cdot \sqrt{(3t^2)^2}\dd{t}= \int_{-1}^1 e^{t^3} \cdot 3t^2\dd{t}.
  \]
\end{Answer}
\end{document}
