\documentclass[8pt,twocolumn]{article}
%\usepackage{amsmath,amssymb,amsthm,amsbsy,amsfonts,mathtools}
\usepackage{amsmath}
\usepackage{amssymb}
\usepackage{amsthm}
\usepackage{physics}
\usepackage{hyperref}
\usepackage{exercise}
\usepackage[makeroom]{cancel}
\usepackage[margin=2em]{geometry}

\usepackage{graphicx}

\usepackage{footmisc}
\DefineFNsymbols{mySymbols}{{\ensuremath\dagger}{\ensuremath\ddagger}\S\P
   *{**}{\ensuremath{\dagger\dagger}}{\ensuremath{\ddagger\ddagger}}}
\setfnsymbol{mySymbols}

\newcommand{\N}{\mathbb{N}}
\newcommand{\R}{\mathbb{R}}
\newcommand{\Z}{\mathbb{Z}}
\newcommand{\Q}{\mathbb{Q}}

\newenvironment{theorem}[2][Theorem]{\begin{trivlist}
\item[\hskip \labelsep {\bfseries #1}\hskip \labelsep {\bfseries #2.}]}{\end{trivlist}}
\newenvironment{lemma}[2][Lemma]{\begin{trivlist}
\item[\hskip \labelsep {\bfseries #1}\hskip \labelsep {\bfseries #2.}]}{\end{trivlist}}
\newenvironment{exercise}[2][Exercise]{\begin{trivlist}
\item[\hskip \labelsep {\bfseries #1}\hskip \labelsep {\bfseries #2.}]}{\end{trivlist}}
\newenvironment{problem}[2][Problem]{\begin{trivlist}
\item[\hskip \labelsep {\bfseries #1}\hskip \labelsep {\bfseries #2.}]}{\end{trivlist}}
\newenvironment{question}[2][Question]{\begin{trivlist}
\item[\hskip \labelsep {\bfseries #1}\hskip \labelsep {\bfseries #2.}]}{\end{trivlist}}
\newenvironment{corollary}[2][Corollary]{\begin{trivlist}
\item[\hskip \labelsep {\bfseries #1}\hskip \labelsep {\bfseries #2.}]}{\end{trivlist}}
\newenvironment{answer}[2][Answer]{\begin{trivlist}
\item[\hskip \labelsep {\bfseries #1}\hskip \labelsep {\bfseries #2.}]}{\end{trivlist}}

\let\emph\relax % there's no \RedeclareTextFontCommand
\DeclareTextFontCommand{\emph}{\bfseries\em}

\setlength{\columnseprule}{0.4pt}
\setlength{\columnsep}{3em}

\usepackage{todonotes}

\author{Yingbo Ma \thanks{Student ID: \tt{16058474}}}
\title{\vspace{-1.cm}Homework 7}
\date{May 28, 2018}

\begin{document}
\maketitle

\begin{Answer}[number=38]
  \begin{proof}
    We are going to proof the proposition by direct proof. We know that every
    line integral along a closed curve is $0$ in such vector field. We have
    \begin{align*}
      \oint_{\Gamma} \bm{F}\cdot \dd{\bm{r}} = \int_{C_1} \bm{F}\cdot \dd{\bm{r}} + \int_{C_2}
      \bm{F}\cdot \dd{\bm{r}} = 0,
    \end{align*}
    where $C_1 + C_2 = C_\Gamma$. Then, we have
    \begin{align*}
      \int_{C_1}\bm{F}\cdot \dd{\bm{r}} &= -\int_{C_2}\bm{F}\cdot \dd{\bm{r}} \\
      \int_{C_1}\bm{F}\cdot \dd{\bm{r}} &= \int_{-C_2}\bm{F}\cdot \dd{\bm{r}},
    \end{align*}
    and $C_1$ and $-C_2$ shares the same end points. Since the curves $C_1$ and
    $-C_2$ are completely arbitrary, we can conclude that $\bm{F}$ is path
    independent. This completes the proof.
  \end{proof}
\end{Answer}

\begin{Answer}[number=39]
  \begin{proof}
    We are going to proof the proposition by direct proof. We know that
    circulation density is
    \begin{align*}
      \lim_{A\to 0}{\frac{1}{\abs{A}}}\oint _{C}\bm{F} \cdot \dd{\bm{r}}
    \end{align*}
    We also know that
    \begin{align*}
      \int_{C_{\bm{a}\to \bm{b}}} \grad f \cdot\dd{\bm{r}} = f(\bm{b}) -
      f(\bm{a}).
    \end{align*}
    Thus, we have
    \begin{align*}
      \oint_{\Gamma} \grad f \cdot\dd{\bm{r}} = 0.
    \end{align*}
    Thus the circulation density is
    \begin{align*}
      &\lim_{A\to 0}{\frac{1}{\abs{A}}}\oint_{\Gamma}\grad f \cdot \dd{\bm{r}} \\
      =&\lim_{A\to 0}{\frac{1}{\abs{A}}}\cancelto{0}{\oint_{C}\grad f \cdot
      \dd{\bm{r}}} = 0.
    \end{align*}
    This completes the proof.
  \end{proof}
\end{Answer}

\begin{Answer}[number=40]
  \begin{enumerate}
    \item \begin{proof}
      We have
      \begin{align*}
        \pdv{Q}{x} &= \frac{1}{x^2+y^2}-\frac{2x^2}{\left(x^2+y^2\right)^2} \\
        \pdv{P}{y} &= \frac{2y^2}{\left(x^2+y^2\right)^2}-\frac{1}{x^2+y^2}.
      \end{align*}
      Thus, we have $Q_x - P_y = 0$. This completes the proof.
    \end{proof}
    \item \begin{proof}
      Let $P(x,y) = -\frac{y}{x^2+y^2}$, and $Q(x,y)=\frac{x}{x^2+y^2}$, thus we
      have $\pdv{P}{y} = \frac{y^2-x^2}{(x^2+y^2)^2}$, and $\pdv{Q}{x} =
      \frac{y^2-x^2}{(x^2+y^2)^2}$. Hence $\pdv{Q}{x}=\pdv{P}{y}$.

      Let $C_1: x(t) = \cos(t), y(t) = \sin(t), 0\le t\le \pi, C_2:x(t) = \cos(t),
      y(t) = \sin(t), t$ from $0$ to $-\pi$. We have
      \[
        \int_{C_1} \bm{F}\cdot \dd{\bm{r}} = \int_0^\pi \frac{\sin^2t +
        \cos^2t}{\sin^2t + \cos^2t} \dd{t} = \int_0^\pi \dd{t} = \pi,
      \]
      and
      \[
        \int_{C_2} \bm{F}\cdot \dd{\bm{r}} = \int_0^{-\pi}\dd{t} = -\pi.
      \]
      Those integrals are not equal. Thus, this field is not conservative.
    \end{proof}
    \item This doesn't contradict the Theorem 16.3.6, since there is a
      singularity at $(0,0)$ in the function $\bm{F}$.
  \end{enumerate}
\end{Answer}

\end{document}
