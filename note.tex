

\section{Iterated integrals}
Intuition for why Fubini's Theorem is true:

Let $R = [a,b]\times [c,d]$, assume $f(x,y)$ is ``nice'' (e.g. continuous is
sufficient)

Fubinii: \[\iint_{R} f(x,y)\dd{A} = \int_a^b\int_c^d f(x,y) \dd{y}\dd{x} =
\int_c^d\int_a^b f(x,y)\dd{x}\dd{y}\]

Reminder: If $\iint_R f(x.y) \dd{A}$ exists, then it is
\[
  \iint_{R}f(x,y)\dd{A} = \overbrace{\lim_{m,n\to \infty} \sum_{i=1}^m \sum_{j=1}^n
                          f(x_{ij}^*, y_{ij}^*)}^{m\cdot n\text{ numbers}} \Delta A
\]
for any choice of sample points $(x_{ij}^*, y_{ij}^*)$: you can choose any
sample point in the $i,j$-th rectangle.

\subsection{Ex:}
Let $R = [0,1] \times [0,1], \quad m=4, n=2$. Let's choose sample points to be
lower-right points in each rectangle. (We can choose anything, could choose
lower-right in the 1st rectangle, middle in the 2nd rectange... As long as
$\iint_R f(x,y) \dd{}A$ exists, the choice doesn't matter as $m,n\to \infty$.)

Riemann sum:
\begin{align*}
  &\iint_{R}f(x,y)\dd{A} =\Big(f(\frac{1}{4},0) + f(\frac{1}{2}, 0) +
                          f(\frac{3}{4},0) + f(1,0)
                          + f(\frac{1}{4},\frac{1}{2})
                          + f(\frac{1}{2}, \frac{1}{2})
                          + f(\frac{3}{4}, \frac{1}{2}) +
                          f(1,\frac{1}{2})\Big) (\frac{1}{4}\cdot \frac{1}{2}) \\
  =& \frac{1}{2}\Big(\underbrace{\sum_{i=1}^4
  f(\frac{i}{4},0)\cdot\frac{1}{4}}_{\text{Riemann sum with }m=4 \text{ for }
  \int_0^1 f(x,0)\dd{x}}\Big) +
  \frac{1}{2}\Big(\underbrace{\sum_{i=1}^4
  f(\frac{i}{4},\frac{1}{2})\cdot\frac{1}{4}}_{\text{Riemann sum with }m=4 \text{ for }
  \int_0^1 f(x,\frac{1}{2})\dd{x}}\Big)
\end{align*}

Intuition for \[\iint_{R} f(x,y) \dd{A} = \int_0^1\underbrace{\int_0^1 f(x,y)
\dd{x}}_{\text{Treat }y \text{as a constant}}\dd{y}\]

\[\sum_{i=1}^4\underbrace{\sum_{j=1}^2 f(\frac{i}{4},
\frac{j-1}{2})\cdot\frac{1}{2}}_{
  \text{Riemann sum for }\int_0^1f(\frac{i}{4},y)\dd{y} \text{ Total: RS for }
  \int_0^1\int_0^1 f(x,y) \dd{y}\dd{x}
}\]

Upshot: Intuition for Fubini's Theorem: $\iint_R f(x,y) \dd{A}$ is a limit of
(finite) Riemann sums. We can arrange those sums so they involve either:
Riemann sums for $f(x,y)$ with $y$ constant or $f(x,y)$ with $y$ constant.

\begin{theorem}
  If $f(x,y)$ is continuous, the $\iint_R f(x,y)\dd{A}$ exists, where $R$ is a
  finite rectangle. Unfortunately, we need to consider discontinuous functions.
\end{theorem}
Why? Say we want to integrate $f(x,y)$ over a shape $S$ which is not a
rectangle.
\paragraph{15.2:} Extend $f(x,y)$ by
\begin{align*}
  F(x,y) = \begin{cases}
    &f(x,y),\quad\text{if } (x,y)\in S \\
    &0,\quad\text{if } (x,y)\not\in S
  \end{cases}
\end{align*}

\section{Application of Fubini's Theorem}
\subsection{Ex:}
Say $g(x),h(y)$ are continuous single-variable functions. Say
$f(x,y)=g(x)h(y)$. Say $R=[a,b]\times[c,d]$. Prove or disprove: $\iint_R f(x,y)
\dd{A} = \int_a^bg(x)\dd{x} \int_c^dh(y)\dd{y}$.

\begin{proof}
  We know
  \begin{align*}
    \iint_R f(x,y) \dd{A} &= \iint_R g(x)h(y) \dd{A} \\
    &=\int_a^b\int_c^d g(x)h(y)\dd{y}\dd{x}\qq{by Fubini's Theorem}
  \end{align*}
  Since $g(x)$ does not depend on $y$, this is equal to
  \[
    \int_a^b g(x)\int_c^dh(y)\dd{y}\dd{x}.
  \]
  Since $\int_c^d h(y)\dd{y}$ does not depend on $x$, this is equal to
  \[
    \int_a^bg(x)\dd{x} \int_c^dh(y)\dd{y} =
    \int_c^dh(y)\dd{y}\int_a^bg(x)\dd{x}.
  \]
\end{proof}
