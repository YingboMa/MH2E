\documentclass[8pt,twocolumn]{article}
%\usepackage{amsmath,amssymb,amsthm,amsbsy,amsfonts,mathtools}
\usepackage{amsmath}
\usepackage{amssymb}
\usepackage{amsthm}
\usepackage{physics}
\usepackage{hyperref}
\usepackage{exercise}
\usepackage[makeroom]{cancel}
\usepackage[margin=2em]{geometry}

\usepackage{graphicx}

\usepackage{footmisc}
\DefineFNsymbols{mySymbols}{{\ensuremath\dagger}{\ensuremath\ddagger}\S\P
   *{**}{\ensuremath{\dagger\dagger}}{\ensuremath{\ddagger\ddagger}}}
\setfnsymbol{mySymbols}

\newcommand{\N}{\mathbb{N}}
\newcommand{\R}{\mathbb{R}}
\newcommand{\Z}{\mathbb{Z}}
\newcommand{\Q}{\mathbb{Q}}

\newenvironment{theorem}[2][Theorem]{\begin{trivlist}
\item[\hskip \labelsep {\bfseries #1}\hskip \labelsep {\bfseries #2.}]}{\end{trivlist}}
\newenvironment{lemma}[2][Lemma]{\begin{trivlist}
\item[\hskip \labelsep {\bfseries #1}\hskip \labelsep {\bfseries #2.}]}{\end{trivlist}}
\newenvironment{exercise}[2][Exercise]{\begin{trivlist}
\item[\hskip \labelsep {\bfseries #1}\hskip \labelsep {\bfseries #2.}]}{\end{trivlist}}
\newenvironment{problem}[2][Problem]{\begin{trivlist}
\item[\hskip \labelsep {\bfseries #1}\hskip \labelsep {\bfseries #2.}]}{\end{trivlist}}
\newenvironment{question}[2][Question]{\begin{trivlist}
\item[\hskip \labelsep {\bfseries #1}\hskip \labelsep {\bfseries #2.}]}{\end{trivlist}}
\newenvironment{corollary}[2][Corollary]{\begin{trivlist}
\item[\hskip \labelsep {\bfseries #1}\hskip \labelsep {\bfseries #2.}]}{\end{trivlist}}
\newenvironment{answer}[2][Answer]{\begin{trivlist}
\item[\hskip \labelsep {\bfseries #1}\hskip \labelsep {\bfseries #2.}]}{\end{trivlist}}

\let\emph\relax % there's no \RedeclareTextFontCommand
\DeclareTextFontCommand{\emph}{\bfseries\em}

\setlength{\columnseprule}{0.4pt}
\setlength{\columnsep}{3em}

\usepackage{todonotes}

\author{Yingbo Ma \thanks{Student ID: \tt{16058474}}}
\title{\vspace{-1.cm}Homework 2}
\date{April 16, 2018}

\begin{document}
\maketitle

\begin{Exercise}[number=10]
  For all $x>1$, let
  \[
    \Gamma(x) := \int_0^\infty t^{x-1}e^{-t}\dd{t}.
  \]
  Using integration by parts and induction, prove that $\Gamma(x)=(x-1)!$ for
  all integers $x\ge1$. You may use without proof that
  $\lim_{t\to\infty}t^ae^{-t}=0$ for all real numbers $a$.
\end{Exercise}

\begin{Answer}[number=10]
\begin{proof}
  I am going to prove this by using induction on the variable $x$.

  \emph{Base case:} If $x=1$, then
  $\int_0^\infty1e^{-t}\dd{t}=\int_{-\infty}^0e^u\dd{u} = e^0-e^{-\infty} = 1$,
  where $u=-t$, and $(1-1)!=1$. Therefore the base case holds.

  \emph{Induction step:} Fix any integer $k\ge 1$, and assume $\int_0^\infty
  t^{k-1}e^{-t}\dd{t}=(k-1)!$, then we want to show $\int_0^\infty
  t^{k}e^{-t}\dd{t}=k!$. We know that $\lim_{t\to\infty}t^ae^{-t}=0$, and
  $0^ae^0=0$ for $a\ne0$. We see that integration by parts yields the following
  expression.
  \begin{align*}
    \int_0^\infty t^{k-1}e^{-t}\dd{t} =
    \underbrace{\cancelto{0}{e^{-t}\cdot\frac{t^k}{k}\Big\vert_0^\infty} +
    \frac{1}{k}\int_0^\infty t^ke^{-t} \dd{t}}_\text{integration by parts}.
  \end{align*}
  By the induction hypothesis, we see that the following holds.
  \begin{align*}
    \int_0^\infty t^{k}e^{-t}\dd{t} &= k\cdot\frac{1}{k}\int_0^\infty t^ke^{-t}
    \dd{t}=k\cdot \int_0^\infty t^{k-1}e^{-t}\dd{t}\\
    &=k\cdot (k-1)!\qq{by the induction hypothesis}\\
    &=k!.
  \end{align*}
  Thus, the induction step holds. Thus we have proved the proposition by
  induction for all $x\ge 1$.
\end{proof}
\end{Answer}

\begin{Exercise}[number=11]
  Proof or disproof: if $R=[a,b]\times[c,d]$ and $f(x,y)=g(x)+h(y)$, where
  $g(x)$ and $h(y)$ are single-variable functions, then
  \[
    \iint_R f(x,y)\dd{A} = \int_a^bg(x)\dd{x} + \int_c^d h(y)\dd{y}.
  \]
\end{Exercise}

\begin{Answer}[number=11]
\begin{proof}
  The statement above is false. A counter example can be established by the
  setting  $[a,b]\times[c,d]=[0,1]\times[0,3], g(x) = 1$, and $h(y)=2$. We
  have
  \begin{align*}
    \iint_R f(x,y)\dd{A} &= \int_a^b\int_c^dg(x) +  h(y)\dd{x}\dd{y} \\
    &= \int_0^1\int_0^3 1 \dd{x}\dd{y} + \int_0^1\int_0^3 2 \dd{x}\dd{y} \\
    &= 1\cdot(3-0)\cdot(1-0) + 2\cdot(3-0)\cdot(1-0) = 9.
  \end{align*}
  However, the value of $\int_a^bg(x)\dd{x} +  \int_c^dh(y)\dd{y}$ is
  \begin{align*}
    \int_a^bg(x)\dd{x} +  \int_c^dh(y)\dd{y}
    &= \int_0^1 1\dd{x} + \int_0^3 2\dd{y} \\
    &= 1\cdot(1-0) + 2\cdot(3-0) = 7.
  \end{align*}
  This completes the disproof, since $7\ne9$.
\end{proof}
\end{Answer}

\begin{Exercise}[number=12]
  Let $g(y)$ be an abbreviation for the function $\int_5^8f(x,y)\dd{x}$. Write
  out the Riemann sum approximation for $\int_7^8 g(y)\dd{y}$ by using two
  intervals and left endpoints. Write out this expression two different ways,
  first using the notation $g(y)$ and then using the notation
  $\int_5^8f(x,y)\dd{x}$.
\end{Exercise}

\begin{Answer}[number=12]
  The expression for the integral in the $g(y)$ form is
  \begin{align*}
    \int_7^8 g(y) \dd{y} &= \sum_{i=1}^2g(y_i^*)(\frac{8-7}{2}) \\
    &=\underbrace{\frac{(g(7)+g(7.5))}{2}}_\text{left points}.
  \end{align*}
  The expression for the integral in the $f(x,y)$ form is
  \begin{align*}
    \int_7^8 g(y) \dd{y} &= \int_7^8\int_5^8f(x,y)\dd{x}\dd{y} \\
    &=
    \sum_{i=1}^2\Big[\sum_{j=1}^1f(x_{ij}^*,y_{ij}^*)(\frac{8-5}{1})\Big](\frac{8-7}{2}) \\
    &= \underbrace{\frac{3\big(f(5,7)+f(5,7.5)\big)}{2}}_\text{left
    points}.
  \end{align*}
\end{Answer}

\begin{Answer}[number=13]
  \emph{15.3.15:} One loop of rose is in the region $S$ that is bounded by
  $-\frac{\pi}{6}\le\theta\le\frac{\pi}{6}$ and $0\le r\le\cos(3\theta)$, thus
  the area is
  \[\int_{-\frac{\pi}{6}}^{\frac{\pi}{6}}\int_0^{\cos(3\theta)}1 r\dd{r}\dd{\theta}.\]

  \emph{15.3.21:} The bottom of the solid is bounded by $x^2+y^2\le1$, which is
  an unit circle that is bounded by $0\le r\le 1$ and $0\le\theta\le2\pi$ in
  the polar coordinate. The top of the solid is bounded by $2x+y+z=4$, which is
  $z=4-2r\cos(\theta)-r\sin(\theta)$ in the polar coordinate. Thus, the volume
  of the solid is given by
  \[
    \int_0^{2\pi}\int_0^1 [4-2r\cos(\theta)-r\sin(\theta)] r \dd{r}
    \dd{\theta}.
  \]

  \emph{15.3.24:} The solid is bounded by $z=1+2x^2+2y^2$ and $z=7$, and their
  intersection is $7=1+2x^2+2y^2$, which is in turn $x^2+y^2=3$. Also, the
  function $z=1+2x^2+2y^2$ in polar coordinate is $z=1+2r^2$. Thus, the volume
  of the solid is given by
  \[
    \int_0^{\frac{\pi}{2}}\int_0^{\sqrt{3}}[7-(1+2r^2)]r\dd{r}\dd{\theta}.
  \]
\end{Answer}

\begin{Exercise}[number=14]
  Let $u = 5x + 3y$.
  \begin{enumerate}
    \item Sketch the region bounded by the lines $x=2, x=3, u=12$ and $u=13$.
      Let $S$ denote this region.
    \item Find, with proof, a formula for $\Delta A$ in terms of $\Delta x$ and
      $\Delta u$. You can use without proof a formula for the area of a
      parallelogram.
    \item Compute $\iint_S (x+y) \dd{A}$ as an iterated integral in terms of
      $x$ and $y$.
    \item Compute $\iint_S (x+y) \dd{A}$ as an iterated integral in terms of
      $x$ and $u$.
  \end{enumerate}
\end{Exercise}

\begin{Answer}[number=14]
  \begin{enumerate}
    \item \includegraphics[width=5cm, height=5cm]{./plot.pdf}
    \item \begin{proof}
        We can calculate $\Delta u$ by using the equality $u=5x+3y$. We have
        \begin{align*}
          u_i &= 5x+3y_i,\\
          u_{i-1} &= 5x+3y_{i-1},\\
          \Delta u = u_i - u_{i-1} &= (5x+3y_i) - (5x+3y_{i-1}) \\
          &= 3y_i - y_{i-1} = 3\Delta y.
        \end{align*}
        We know that $\Delta A$ is equal to $\Delta x\cdot\Delta y$. Thus we
        have
        \[
          \Delta A = \Delta x\cdot \Delta y = \Delta x\cdot\frac{\Delta u}{3}.
        \]
        The proof is completed.
  \end{proof}
  \item Integrating the integral $\iint_S (x+y) \dd{A}$ in terms of $x$ and $y$
    is bounded by the lines $x=2, x=3, y=\frac{12-5x}{3}$ and
      $y=\frac{13-5x}{3}$, since we know that $u=5x+3y$. We have
      \begin{align*}
        &\int_2^3\int_{(12-5x)/3}^{(13-5x)/3} x+y \dd{y}\dd{x} =
        \int_2^3 xy+\frac{y^2}{2}\Big\vert_{y=(12-5x)/3}^{y=(13-5x)/3}\dd{x} \\
        =& \int_2^3 \frac{1}{3} + \frac{25-10x}{18}\dd{x} = \big( \frac{x}{3}
        \cdot \frac{x^2}{2} + \frac{25x}{18} - \frac{10}{18}\cdot\frac{x^2}{2}
        \big) \\
        =& \frac{5}{6} + \frac{25}{18} - \frac{25}{36} = \frac{5}{6}.
      \end{align*}
    \item Integrating the integral $\iint_S (x+y) \dd{A}$ in terms of $x$ and
      $u$ is bounded by the lines $x=2, x=3, u=12$ and $u=13$. We have to
      transform the integrand $x+y$ to a form that is only depend on $u$ and
      $x$ by the relation $u=5x+3y$. We have
      \[
        x+y = x + \frac{u-5x}{3} = \frac{u-2x}{3}.
      \]
      Thus the integral is now
      \begin{align*}
        &\int_2^3\int_{12}^{13}\frac{u-2x}{3}
        \underbrace{\frac{1}{3}\dd{u}\dd{x}}_\text{from part (2)} =
        \frac{1}{9}\int_2^3(\frac{u^2}{2}-2xu)\Big\vert_{u=12}^{u=13}\dd{x} \\
        =& \frac{1}{9} \cdot (\frac{u^2}{2}\vert_{u=12}^{u=13}\cdot
        x\vert_{x=2}^{x=3} -
        2\cdot \frac{x^2}{2}\vert_{x=2}^{x=3}\cdot u\vert_{u=12}^{u=13}) \\
        &= \frac{25-10}{18} = \frac{5}{6}.
      \end{align*}
  \end{enumerate}
\end{Answer}

\begin{Answer}[number=15]
  The ``proof'' is incorrect since $\sqrt{2}$ cannot represent all irrational
  numbers, and a special case is not enough to support proof by contradiction.
  In addition, a simple counter example
  \[\underbrace{\sqrt{2}}_\text{irrational} +
  \underbrace{(-\sqrt{2})}_\text{irrational} = \underbrace{0}_\text{rational}.\]
  is enough to disprove the proposition.
\end{Answer}

\begin{Exercise}[number=16]
  What is the domain of the function $\frac{1}{\sqrt{-\log(u)}}$? Compute that
  \[
    \int_0^1 \frac{\dd{u}}{\sqrt{-\log(u)}} = \sqrt{\pi}
  \]
\end{Exercise}

\begin{Answer}[number=16]
  The domain of the function $\frac{1}{\sqrt{-\log(u)}}$ is $(0,1)$. The
  integral
  \[
    \int_0^1 \frac{\dd{u}}{\sqrt{-\log(u)}} = \sqrt{\pi}
  \]
  can be computed by using calculate the integral $\int_0^\infty
  e^{-x^2}\dd{x}$ first.
  \begin{align*}
    &\int_0^\infty e^{-x^2}\dd{x} = \sqrt{(\int_0^\infty e^{-x^2}\dd{x})^2} \\
    &= \sqrt{\int_0^\infty e^{-a^2}\dd{a}\int_0^\infty e^{-b^2}\dd{b}}
    = \sqrt{\int_0^\infty\int_0^\infty e^{-(a^2+b^2)}\dd{a}\dd{b}} \\
    &= \sqrt{\int_0^\frac{\pi}{2}\int_0^\infty e^{-r^2} \cdot r
    \dd{r}\dd{\theta}}
    = \sqrt{\frac{\pi}{2}\int_0^\infty e^{-r^2} \cdot r \dd{r}} \\
    &= \sqrt{\frac{\pi}{4}\int_{-\infty}^{0} e^c \dd{c}},\quad c =
    -r^2, \dd{c} = -2\dd{r}  \\
    &= \sqrt{\frac{\pi}{4}\lim_{x\to-\infty}e^f\vert^0_x} =
    \sqrt{\frac{\pi}{4}\cdot 1} = \frac{\sqrt{\pi}}{2}
  \end{align*}
  Then, we can calculate $\int_0^1 \frac{\dd{u}}{\sqrt{-\log(u)}}$ via
  substitution.
  \begin{align*}
    \int_0^1 \frac{\dd{u}}{\sqrt{-\log(u)}} &= \int_\infty^0
    -\frac{e^{-y}}{\sqrt{y}} \dd{y},\quad y=-\log(u), \dd{y} = -e^y\dd{u} \\
    &= \int_0^\infty x^{-1} e^{-x^2} 2x \dd{x},\quad y=x^2,
    \dd{y}=2x\dd{x} \\
    &= 2\int_0^\infty e^{-x^2} \dd{x} = 2\frac{\sqrt{\pi}}{2} = \sqrt{\pi}
  \end{align*}
\end{Answer}

\end{document}
